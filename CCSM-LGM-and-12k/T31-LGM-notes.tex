\documentclass[authoryear,round,12pt]{article}
\usepackage{fancybox}
\usepackage{color}
\usepackage{natbib}
\usepackage{graphicx}
\usepackage{setspace}
\usepackage{amsmath}
\usepackage{pslatex}
\setlength{\oddsidemargin}{0pt}
\setlength{\evensidemargin}{0pt}
\setlength{\textwidth}{6.5in}
\setlength{\textheight}{9.2in}
\setlength{\topmargin}{-0.55in} % pdflatex
\begin{document}


\begin{center}
{\Large \bf Working notes for creating CCSM3 T31 LGM run}

\medskip

Ian Eisenman, 2005 (eisenman@post.harvard.edu)

\medskip

Thanks to help from Bette Otto-Bliesner, Esther Brady, Bruce Briegleb,
Cecilia Bitz

\end{center}

\medskip

\noindent This document and all files I edited for this run can be downloaded from my website or obtained by emailing me. They are also available on NCAR mass storage at\\
\verb+/EISENMAN/working-notes/T31-LGM.tar.gz+\\
All files are in a directory called T31-LGM/, and paths to files in the directory are mentioned in this document.

\medskip

\section{Introduction}

These are notes I put together while setting up a low resolution CCSM3
simulation with last glacial maximum (LGM) forcing, which was presented in \citet{Eisenman-Bitz-Tziperman-2007:rain}.  This experiment
is based closely on a previous high resolution LGM experiment
\citep{Otto-Bliesner-Brady-Clauzet-et-al-2006:last}, as well as a
previous low resolution run with 1780 forcing
\citep{Otto-Bliesner-Tomas-Brady-et-al-2006:climate}. The high
resolution run uses T42 atmosphere/land and gx1v3 ocean/ice; the low
resolution runs use T31 atmosphere/land and gx3v5 ocean/ice. At times
in these notes (e.g., the title) I simply use T31 or T42 to refer to
the resolution, whether discussing just the ocean or the whole coupled
run.

Compared to 1780 (the paleoclimate control run), the LGM has different
sea level (simulated here by changing the coastline but leaving the
bathymetry otherwise largely unaltered), ice sheets (affecting the
land topography and surface type), greenhouse gases, and insolation.
Section \ref{sec:data} describes the process of taking the
topography/bathymetry and land surface type datasets from the ICE-5G
reconstruction \citep{Peltier-2004:global} and creating the necessary
data files for the model. Note that I used some files already created
for the T42 LGM run. Section \ref{sec:namelists} describes setting up
the run and making the necessary changes in the namelists.

A document I have found very helpful in this work is ``Procedures for
Generating Input files for CCSM2'', written by Esther Brady, Steve
Yeager, Christine Shields, Sam Levis, and Bette Otto-Bliesner. I've
copied the file to T31-LGM/doc/SetupCCSM3.pdf.

\subsection{Computers}

Although it would be nice to do this all on one computer if it had the
necessary FORTRAN libraries, the makefiles were configured correctly,
etc, I ended up using 5 computers for this work: bluesky (AIX),
tempest (IRIX64), neva (SunOS), poorman (Linux), and my laptop (XP; I
used it for Matlab). I try to indicate in the notes below which
computer I used for various jobs. I worked primarily on bluesky and
copied files to other computers as necessary, ran jobs, and then
copied the output files back to bluesky.

\section{Creating input data for model}
\label{sec:data}

\subsection{ATM}

Need to create LGM surface BC (topography). This is contained in the
...cam2.i...nc init file. I used the T31 1780 run (b30.105) atm init
file from year 100, changing the topography to be PD plus the
difference between LGM and PD in the Peltier ICE-5G
reconstructions. This was all done with the fortran program
definesurf, in \verb+T31-LGM/atm-definesurf/+. The program is
described in the README file, and my use of it is documented in
definesurf.readme. The final product was the file
\verb+b30.105.cam2.i.0100-01-01.LGM_c050523.nc+

Unfortunately, definesurf doesn't change the variable
LANDM{\_}COSLAT. I changed this with the ncl program in
\verb+T31-LGM/atm-definesurf/landmcoslat/+, described in the .readme
file, and created the new file
\verb+b30.105.cam2.i.0100-01-01.LGM_c050615.nc+

Note that while I am working primarily on bluesky, definesurf was run
on tempest and fix{\_}landmcoslat was run on neva, as explained in the
.readme files.

\subsection{LND}

I used the data already prepared for b30.104, since the land accepts
high resolution surface-type datasets and interpolates them to T31 at
the start of the run. The files are described in section
\ref{sec:namelists}.

\subsection{OCN}

\subsubsection{Bathymetry}

Needed to create ocean bathymetry files:

KMT (bin) file contains depth(lat,lon) measured in number of
boxes. KMT=0 over land, so this contains the coast.

Region mask (bin) file contains basin(lat,lon), where the basin is an
integer corresponding to which ocean it is. Negative integers indicate
marginal seas.

Region ids (text) file contains the names of each numbered basin and
the location and area over which each marginal sea imposes it's
salinity balance.

The first two files are created by first zeroing all shallow water and
making a few corrections based on region mask (``method 1'') to
approximate LGM ocean starting with PD ocean. This is calculated using
\verb+T31-LGM/ocn-kmt/makekmtgx3v5.f+. Next, the file is visually
compared in matlab with the Peltier ICE-5G reconstruction, and ocean
points are manually selected to be changed to land (``method 2''). The
matlab routine, \verb+T31-LGM/ocn-kmt/matlab/KMT_gx3v5_compare.m+,
creates a text file identifying the grid boxes selected. This is read
by the Fortran program makekmtgx3v5.f, and final versions of the kmt
and region mask files are created. This entire procedure is described in
\verb+T31-LGM/ocn-kmt/makekmtgx3v5.readme+.

Next, I examined the locations in the ocean for the marginal sea
balances and slightly adjusted the ocean target for the Caspian sea in
\verb+gx3v5_region_ids+; I also re-numbered some of the regions
because of dried up seas. I saved this altered file as
\verb+gx3v5_region_ids_LGM+.

\subsubsection{Initial condition}

I also needed to create T-S initial condition based on the climatology
of b30.104 at a fairly late (i.e., spun up) time in the run. Hence, I
needed to interpolate from a gx1v3 (netcdf) hist file to gx3v5 (bin)
init file. I did this with \verb+T31-LGM/ocn-ic/lev2grid.csh+, which
requires the program regrid (compiled to work on tempest) as well as
Matlab with the netcdf toolbox (I used my laptop). My usage of the
program is described in \verb+T31-LGM/ocn-ic/lev2grid.readme+. The
final product was the file \verb+LGM_jan_ic_gx3v5_Y0290-0299.ieeed+.

\subsection{ICE}

This module just takes the KMT file made for the ocean.

This run, like b30.104, will start from the default sea ice initial
condition.

\subsection{CPL}

Need to create 4 mapping files:

\begin{verbatim}
 map_T31_to_gx3LGM_aave_da.nc
 map_T31_to_gx3LGM_bilin_da.nc
 map_gx3LGM_to_T31_aave_da.nc
 map_r5_to_LGMgx3.nc
\end{verbatim}

The last one, which maps from the runoff grid to the ocean grid, takes
the longest to make. This appears to be mostly because of smoothing -
1 runoff point needs to map to many ocean points.

Procedure:
\begin{description}
 
\item[(1)] Create mapping files between T31 atm and gx3v5 LGM
  ocean: Use\\
  \verb+T31-LGM/cpl-mapping/mk_remap.csh+; this script, and my usage
  of it, are documented in \verb+T31-LGM/cpl-mapping/mk_remap.readme+.

\item[(2)] Correct mapping errors (related to singularities at poles).
  Uses the idl routine\\
  \verb+T31-LGM/cpl-mapping/correct_map_errors.run+; my usage is
  documented in\\
  \verb+T31-LGM/cpl-mapping/correct_map_errors.readme+. This is done
  only for aave (i.e., conservative) mappings. The maps I'd created
  had no errors since 90.0 north was changed to 89.9999 in
  T31\_040122.nc.

\item[(3)] Create initial map from runoff grid to gx3v5 LGM ocean:
  0.5x0.5 runoff grid to gx3v5 LGM ocean grid mapping is created with
  \verb+T31-LGM/cpl-mapping/mk_runoff_remap.csh+; usage is explained
  in \verb+T31-LGM/cpl-mapping/mk_runoff_remap.readme+.

\item[(4)] Smooth runoff map (this is most computationally expensive
  step): Need to (a) correct, (b) smooth, and (c) check map. The
  smoothing is to avoid making sea surface salinity get too perturbed
  in a single box. This is all done in one fell swoop in a FORTRAN
  program compiled on bluesky and then submitted to the cluster; a
  full
  description is at\\
  \verb+T31-LGM/cpl-mapping/map_runoff/main.F90.full.readme+.

\end{description}

\section{Creating a new experiment and modifying namelists}
\label{sec:namelists}

The relevant scripts from the new experiment are copied in\\
\verb+T31-LGM/scripts/T31.LGM.test/+

\subsection{creating default low resolution CCSM run}
On bluesky
\begin{verbatim}
cd /fis/cgd/cseg/csm/collections/ccsm3_0_rel04/scripts
./create_newcase -case $HOME/scripts/T31.LGM.test -res T31_gx3v5 -mach bluesky
cd $HOME/scripts/T31.LGM.test
\end{verbatim}
Edit \verb+env_conf+ to change the case string to T31 LGM test run
\begin{verbatim}
./configure -mach bluesky
\end{verbatim}

\subsection{Copying over relevant data files}

I copied the files created in section \ref{sec:data} to the SourceMods
directory of the new experiment, following the convention used in
b30.104 (even though these aren't really modifications to the source
code). All the file copying is documented in \verb+SourceMods/readme+.

\subsection{Namelist modifications}

Note that by default data files are often copied to the run directory
(/ptmp/user/expt) using the script ccsm\_getinput, but apparently the
script ccsm\_cpdata should be used instead for user-specified data
files. I ended up modifying the ccsm\_cpdata script to avoid errors
when the model was built twice - see note in section
\ref{sec:namelists}\ref{sec:building} below.

These modifications were made based on using diff (or ediff) with
previous experiments. I got the previous run scripts from the
following locations:

\begin{verbatim}
# b30.004=T42_PD; 100.02=T42_1780; 104[w]=T42_LGM; 031=T31_PD; 105=T31_1780;
set cgd=/fis/cgd/cseg/csm/runs/ccsm3_0
set zav=/home/bluesky/zav/ccsm_runs
cp -r $zav/b30.104 $zav/b30.104w $cgd/b30.105 .
# note that /fis/cgd/ccr/paleo/ccsm_runs should have same scripts as in zav
\end{verbatim}

\subsubsection{ATM}

For LGM, need to change topography (note that the atmosphere has
pressure levels near the surface and height levels higher up; PHIS
specifies surface geopotential; SGH specifies orography standard
deviation, used for gravity waves) and surface pressure (PS; note that
the land surface during the LGM is at a higher elevation on average
compared the sea level because of the ice sheets, but the pressure
should be the same since the atmosphere has the same mass; this subtle
point is addressed in b30.104, and hence also here, because it is
required by PMIP). Also need to change ozone and aerosols from default
PD values to 1780 values, and change CO$_2$, CH$_4$, N$_2$O to LGM
values.

\verb+Buildnml_Prestage/cam.buildnml_prestage.csh+: Give location of
1780 aerosol and ozone data as in b30.105. Change initial condition
(datinit), which includes topography, to LGM file generated in Section
\ref{sec:data}. I took the fincl1 and fincl2 fields - extra variables
to include in primary and secondary history files - from
b30.104w. Mentioned ozncyc (true to assume 1 year ozone data rather
than longer dataset) and scenario\_carbon\_scale take their default
value; presumably unnecessary but following b30.104/b30.104w/b30.105.
Set solar constant as in b30.104/b30.104w/b30.105, and used greenhouse
forcings as in b30.104.

\subsubsection{LND}

\verb+Buildnml_Prestage/clm.buildnml_prestage.csh+: Indicated datasets
built based on Peltier ICE-5G for b30.104 for glacier, raw inland
water, and plant functional type distributions. Set fsurdat='' so that
it will be generated during first run, after which I will copy it from
execute directory to SourceMods/src.clm/surface-data.096x048.LGM.nc
and specify it in fsurdat.

\verb+SourceMods/src.clm/mksrfdatMod.F90+: This file corrects default
to put wetlands on glaciers since they weren't specified otherwise
(b30.104).  Also makes new coast area nearest neighbor type rather
than wetland (b30.104w).

Note that I used present day (default) runoff rather than create a
file based on the new topography. b30.104/b30.104w did the same -
didn't seem worth the headache. Also note that in b30.105 (b30.031
hybrid), finidat is specified. This is initialization data for
prognostic land model variables, used by hybrid runs but not b30.104
or me.

\subsubsection{OCN}

\verb+Buildnml_Prestage/pop.buildnml_prestage.csh+: Set initial
condition T-S distribution file, LGM topography/coastline, and mask
file (tells whether it's a marginal sea, as well as name of basin for
diagnostics).

\subsubsection{ICE}

\verb+Buildnml_Prestage/csim.buildnml_prestage.csh+: Include
indication of LGM topography/coastline file as in b30.104. Note that
by default lower ice albedos, specified here, are used for T31 runs
compared with T42 (for better fit with PD observations).

\subsubsection{CPL}

Buildnml{\_}Prestage/cpl.buildnml{\_}prestage.csh: Change orbital
forcing to orb{\_}year = -19050 and indicate new mapping files.

\subsection{Building and submitting run}
\label{sec:building}

On bluesky, having made all the modifications described above, try to
build the model.
\begin{verbatim}
./T31.LGM.test.bluesky.build
\end{verbatim}

Note that when a run is started (if CONTINUE\_RUN is FALSE), the model
is automatically built. I replaced a handful of lines in the namelist
prestage scripts that had ccsm\_getinput with ccsm\_cpdata (allows you
to specify exactly where data file is rather than searching for it),
as described above and in the readme files. While ccsm\_getinput will
not copy a file if it already exists in the target directory,
ccsm\_cpdata will. This causes errors due to read-only file
permissions when the model is re-built at the start of a run. I could
have just skipped the build step, so the model would only be built
once, but instead I copied ccsm\_cpdata to a local folder and edited
it not to overwrite files that already exist (to avoid any other
errors). See Buildnml{\_}Prestage/Tools/.

In env\_run, set length of time for run (STOP\_OPTION, STOP\_N) and
number of times to resubmit run (RESUBMIT), and submit run to cluster.
\begin{verbatim}
llsubmit T31.LGM.test.bluesky.run
\end{verbatim}

\section{Debugging}

\begin{center}
Thanks to further help from Brian Kauffman, Esther Brady, Cecilia Bitz,\\ 
Bruce Briegleb, Bette Otto-Bliesner
\end{center}

The experiment initially had few bugs, which were fixed as follows:
\begin{description}

\item[(1)] Model crashed with LGM CAM datinit file (topography): This
  was apparently because CAM runs close to instability with the
  default 30min time step. I reduced the time step to 10min.

\item[(2)] Model crashed with LGM POP T-S initial condition: This was
  because the TS ic binary file was written in the native machine
  format, which was apparently different on the machine I used to
  create the file from on the machine where the script was originally
  run. I changed lev2grid.csh to explicitly specify the binary format
  (ieee-be) in the Matlab script it creates.

\item[(3)] Model crashed with LGM KMT/regmask/mappings: This was
  because, in all my copying of files between computers, I
  accidentally built the coupler mapping files with the penultimate
  iteration of the KMT file, which differed from the final KMT file at
  several grid points. So the model saw one ocean coastline in the
  KMT/regmask files, and a second one in the coupler maps. This led
  the coupler to mapvalues of 1e30 (missing value) from the ocean
  (outside the ocean region) to the atmosphere.

\end{description}

To address bug (3), I ran the model with several source code
modifications, asking it to print a coupler history file (NetCDF)
earlier in the initialization and to include several more fields in
the file. There was an issue that a run can spend about 4-25 hours in
the queue before running, which was resolved by using a partial node
and allowing sharing in the 32-node cluster (this probably made the
model run a lot slower, but jobs spent almost no time waiting in the
queue). I wrote a shell script to create new cases of this type, as I
was running a lot of debug experiments. I also ran the model with
horizontally uniform T-S i.c. (based on a Levitus T-S profile); with
no sea ice initially; and outputting history files every time
step. These modifications are copied in\\
\verb+T31-LGM/debug/T31.LGM.debug_mods/+

I used several Matlab scripts to examine the setup files. These are in\\ 
\verb+T31-LGM/debug/+

\bibliographystyle{apalike}
\bibliography{all}

%\begin{flushright}
%  {\it Printed \today}
%\end{flushright}

\end{document}
